\documentclass[a4paper,12pt,oneside]{scrartcl}

% Setup page
\usepackage[top=2.0cm, bottom=2.0cm, left=3cm, right=1.5cm, footskip=0.5cm]{geometry}
\usepackage{epsfig}

% Base packages
\usepackage{fontspec}
\usepackage{xunicode}
\usepackage{xltxtra}
\usepackage{comment}
\usepackage{amsfonts}
\usepackage{amsmath}
\usepackage{longtable}
\usepackage{csquotes}
\usepackage{setspace}
\usepackage{placeins}
\usepackage{float}
\usepackage{listings}

%setup listings
\lstset{language=C++,
                %basicstyle=\ttfamily,
                keywordstyle=\color{blue},%\ttfamily,
                stringstyle=\color{red},%\ttfamily,
                commentstyle=\color{green},%\ttfamily,
                morecomment=[l][\color{magenta}]{\#}
}

% Setup Russian hyphenation
\usepackage{polyglossia}
\setdefaultlanguage[spelling=modern]{russian} % for polyglossia
\setotherlanguage{english} % for polyglossia
\defaultfontfeatures{Scale=MatchLowercase, Mapping=tex-text}

% Setup fonts
\newfontfamily\russianfont{CMU Serif}
\setromanfont{CMU Serif}
\setsansfont{CMU Sans Serif}
\setmonofont{CMU Typewriter Text}

\setstretch{1.25}

% Be able to include PNG and PDF as images
\usepackage{graphicx}

% Be able to insert hyperlinks
\usepackage{hyperref}
\hypersetup{colorlinks=true, linkcolor=black, filecolor=black, citecolor=black, urlcolor=black , pdfauthor=Grigory Rechistov <grigory.rechistov@phystech.edu>, pdftitle=Шаблон для написания диплома в LaTeX}
\usepackage{url}

% Misc optional packages
\usepackage{footnpag}
\usepackage{indentfirst}
\usepackage{underscore}
\usepackage{amsthm}
\usepackage{enumitem} % continue enumeration 
\usepackage{mdwlist} % compact itemize lists environment
\usepackage{mdwlist} % compact itemize lists environment
%\renewcommand{\labelitemi}{--}  % Use endash for itemized lists
\usepackage[hang,flushmargin]{footmisc} % correct indent for footnotes


% Attach TikZ/PGF system to be able to draw vector plots.
% \usepackage{tikz}
% \usetikzlibrary{shapes, calc, arrows, fit, positioning, decorations, patterns, decorations.pathreplacing,chains, snakes}

% A new command to mark not done places
\newcommand{\todo}[1][Напиши меня]{{\color{red}TODO\ #1}}

\begin{document}

\begin{titlepage}
\thispagestyle{empty}

{\centering\bfseries\upshape
Министерство образования и науки Российской Федерации
\par}

{\centering
Федеральное государственное автономное образовательное учреждение высшего профессионального образования 
<<Московский физико{}-технический институт (государственный университет)>>\par}

{\centering
Факультет радиотехники и кибернетики \par}

{\centering 
Кафедра <<Инфокоммуникационные системы и сети>>
\par}

%\endhead

\bigskip

\bigskip

\vfill

{\centering\large
Козырский Николай Михайлович
\par}
   
{\centering\Huge

Сравнение ядерного алгоритма k-средних с методом Уорда в задачах иерархической кластеризации на графах

\par}

\bigskip

{\centering\bfseries
Бакалаврская диссертация

Направление подготовки: 030301 Прикладные математика и физика
Бакалаврская программа: 010674 Телекоммуникационные сети и системы 

\par}

\vfill

\begin{tabular}{lll}\\
Заведующий кафедрой  & ______________ & /______________/ \\
                     &                &                  \\
Научный руководитель & ______________ & /______________/ \\
                     &                &                  \\
Студент              & ______________ & /______________/ \\
\end{tabular}

\vfill

{\centering
г. Москва 

2018
\par}

\end{titlepage}


\cleardoublepage


\tableofcontents
\listoffigures
\listoftables

\cleardoublepage


\section{Введение}\label{sec:intro}

Всем привет


\cleardoublepage
\begin{thebibliography}{99}
\bibitem{B1} L. A. Belady. A study of replacement algorithms for a virtual-storage computer. In IBM Systems journal, pages 78–101, 1966.

\bibitem{B2} M. K. Qureshi, A. Jaleel, Y. N. Patt, S. C. Steely, and J. Emer. Adaptive insertion policies for high performance caching. In ISCA-34, 2007.

\bibitem{B3} K. So, R. N. Rechtschaffen.  Cache   Operations   by   MRU   Change. In IEEE Transaction Computers, Vol. 37, N. 6, IEEE Computer Society, pp. 700-709, 1988. 

\bibitem{B4}  A. Gonzalez, C. Aliagas, and M. Valero. A data cache with multiple caching strategies tuned to different types of locality. In ICS-9, 1995.

\bibitem{B5} N. Megiddo and D. S. Modha. ARC: A self-tuning, low overhead replacement cache. In Proceeding of the 2nd USENIX Conference on File and Storage Technologies, 2003

\bibitem{B6} Y. Smaragdakis et al. The EELRU adaptive replacement algorithm. Performance Evaluation, 53(2):93–123, 2003

\bibitem{B7} T. Austin, D. Ernst and E. Larson, "SimpleScalar: An Infrastructure for Computer System Modeling," in Computer, vol. 35, no. , pp. 59-67, 2002. 

\bibitem{B8} J. Renau, B. Fraguela, J. Tuck, W. Lui, M. Prvulovic, L. Ceze, S. Sarangi, P. Sack, K. Struss, and P. Montesinos. Simulator for cmp architecture.

\end{thebibliography}

\addcontentsline{toc}{section}{Список литературы}
%=====================================================
\cleardoublepage

\end{document}